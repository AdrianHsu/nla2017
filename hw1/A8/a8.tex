\documentclass[12pt, a4paper]{article}
\usepackage[utf8]{inputenc}
\usepackage{amsmath}
\usepackage{amsfonts}
\newtheorem{theorem}{Theorem}
\usepackage{listings}
\usepackage{xcolor}

% AH: used for row vector
\usepackage{xparse}
\ExplSyntaxOn
\NewDocumentCommand{\rvect}{m}
 {
  \seq_set_split:Nnn \l_tmpa_seq { , } { #1 }
  \begin{bmatrix}
  \seq_use:Nn \l_tmpa_seq { & }
  \end{bmatrix}
 }
\ExplSyntaxOff

\title{NLA2017, HW\#1, Problem Set A}
\author{PIN-CHUN, HSU}
\begin{document}
\maketitle{}

{\large \textbf{A8.} Sec 7.6: 15} \\
\\
\textbf{a.} Show that an \textit{A}-orthogonal set of nonzero vectors associated with a positive definite matrix is linearly independent.
\\
\\
\textit{pf.} prove by contradiction, suppose that any \textit{A}-orthogonal set $S = \{\mathbf{v}_1, \mathbf{v}_2...\mathbf{v}_n\}$ is linear dependent; that is, there exists a finite number of distinct vectors $V = \{\mathbf{v}_1, \mathbf{v}_2, ..., \mathbf{v}_k\}$ in $S$, $k < n$, and scalars $a_1, a_2, ..., a_k$ not all zero, such that $$ a_1 \mathbf{v}_1 + ... + a_k \mathbf{v}_k = 0$$ 
If not all of the scalars are zero, then at least one is non-zero, WLOG, say $a_{1}$, in which case this equation can be written in the form $$\mathbf{v}_1 = -\cfrac{a_2}{a_1}\mathbf{v_2}-...-\cfrac{a_k}{a_1}\mathbf{v_k}$$
Since that $$\mathbf{v}_1^{t}A\mathbf{v}_j = 0, j \neq 1$$and choose $\mathbf{v}_j$ which is in the set $V$, we get $$(-\cfrac{a_2}{a_1}\mathbf{v_2}-...-\cfrac{a_k}{a_1}\mathbf{v_k})^{t}A\mathbf{v}_j = 0$$ that is, $$\left((-\cfrac{a_2}{a_1}\mathbf{v_2})^{t}+...+(-\cfrac{a_k}{a_1}\mathbf{v_k})^{t}\right)A\mathbf{v}_j = 0$$ 
by definition of A-orthogonal set, since that $$(-\cfrac{a_2}{a_1}\mathbf{v_l})^{t}A\mathbf{v}_j = 0$$ where $l = 2, 3, ...k$ except $j$, therefore the equation can be reduced to $$(-\cfrac{a_j}{a_1}\mathbf{v_j})^{t}A\mathbf{v}_j = 0$$ remove the scalar, we get $$\mathbf{v_j}^{t}A\mathbf{v}_j = 0$$ but we know that $A$ is a positive definite matrix, i.e. $\mathbf{v}^{t}A\mathbf{v} > 0$ for all $\mathbf{v} \neq \mathbf{0}$, which causes a contradiction. $\Box$
\\
\\
\\
\textbf{b.} Show that if $\{\mathbf{v}^{(1)}, \mathbf{v}^{(2)},... , \mathbf{v}^{(n)}\}$ is a set of \textit{A}-orthogonal nonzero vectors in $\mathbb{R}$ and $\mathbf{z}^{t}\mathbf{v}^{(i)} = 0$, for each $i = 1,2,... ,n$, then $\mathbf{z} = \mathbf{0}$.
\\
\\
\textit{pf.} since that $\mathbf{z}^{t}\mathbf{v}^{(i)} = 0$ for every $i$, we can concatenate every vectors into a matrix as its \textbf{column vectors}, that is,$$\mathbf{z}^{t}\rvect{\mathbf{v}^{(1)}, \mathbf{v}^{(2)},... , \mathbf{v}^{(n)}} = \mathbf{0}_{1 \times n}$$ and then we take transpose, $$\rvect{\mathbf{v}^{(1)}, \mathbf{v}^{(2)},... , \mathbf{v}^{(n)}}^{t}\mathbf{z} = \mathbf{0}_{n \times 1}$$ we denote the n-by-n matrix $\rvect{\mathbf{v}^{(1)}, \mathbf{v}^{(2)},... , \mathbf{v}^{(n)}}^t = V_{n \times n}$, that is, $$V\mathbf{z} = \mathbf{0}$$ which is a \textbf{homogeneous linear system} of the form $A\mathbf{x} = \mathbf{0}$. we can use the following theorem:
\begin{theorem}
    Let $A$ be an $n \times n$ matrix, if $|A| \neq 0$, then the homogeneous linear system $A\mathbf{x} = \mathbf{0}$ has only the trivial solution, $\mathbf{x} = \mathbf{0}$.
\end{theorem}
since that $V_{n \times n} = \rvect{\mathbf{v}^{(1)}, \mathbf{v}^{(2)},... , \mathbf{v}^{(n)}}^{t}$ is formed by the $A$-orthogonal set, by problem \textbf{a.} we already knew that the set is linearly independent, so that $$|V_{n \times n}^t| \neq 0 $$ and also by the properties of determinent as below, $$|A| = |A^{t}|$$ therefore we get,$$| V^{t} | = | V | \neq 0 $$
therefore by \textbf{Theorem 1}, we discovered that $V\mathbf{z} = \mathbf{0}$ has only the trivial solution, that is , $\mathbf{z} = \mathbf{0}$. $\Box$   



\end{document}
